\documentclass[journal,11pt,twocolumn]{IEEEtran}

\usepackage{amsmath}
\providecommand{\pr}[1]{\ensuremath{\Pr\left(#1\right)}}
\providecommand{\cbrak}[1]{\ensuremath{\left\{#1\right\}}}
\newcommand*{\permcomb}[4][0mu]{{{}^{#3}\mkern#1#2_{#4}}}
\newcommand*{\comb}[1][-1mu]{\permcomb[#1]{C}}

\title{Assignment 2}
\author{Velma Dhatri Reddy \\ \normalsize AI21BTECH11030 \\ \vspace*{10pt} \Large ICSE 2019 Grade 12}
\begin{document}
\maketitle
\textbf{Question 10:}
Bag A contains 4 white balls and 3 black balls, while Bag B contains 3 white balls and 5 black balls. Two balls are drawn from Bag A and placed in Bag B. Then, what is the probability of drawing a white ball from Bag B?

\textbf{Solution:} Let $X=\cbrak{0,1}$ be a random variable representing the bags and let $Y=\cbrak{0,1}$ be a random variable represent the colour of the ball.

See Tables 
	\eqref{table:table-1}
	and 
	\eqref{table:table-2} 
\begin{table}[ht!]
	\begin{tabular}{|l|c|}

\hline
\textbf{Event} & \textbf{Description} \\
\hline
$X = 0$ &  Ball is drawn from bag A \\
\hline
$X = 1$ &  Ball is drawn from bag B  \\
\hline
$Y = 0$ &  Colour of the ball is white \\
\hline
$Y = 1$ &  Colour of the ball is black\\
\hline
$Z= 0$ &  Both balls drawn are white \\
\hline
$Z= 1$ &  Both balls drawn are black \\
\hline
$Z = 2$ &  One ball is black and other is white \\
\hline
\end{tabular}
	\vspace*{5pt}
\caption{}
	\label{table:table-1}
\end{table}
\begin{table}[ht!]
	\begin{tabular}{|l|c|}

\hline
\textbf{Probability} & \textbf{Value} \\
\hline
\pr{Z=0|X=0} & $\dfrac{\comb{4}{2}}{\comb{7}{2}} = \dfrac{2}{7}$\\
\hline
\pr{Z=1|X=0} & $\dfrac{\comb{3}{2}}{\comb{7}{2}} = \dfrac{1}{7}$\\
\hline
\pr{Z=2|X=0} & $\dfrac{\comb{4}{1}\times\comb{3}{1}}{\comb{7}{2}} = \dfrac{4}{7}$\\
\hline
\pr{Y=0|Z=0,X=1} & $\dfrac{5}{10}$\\
\hline
\pr{Y=0|Z=1,X=1} & $\dfrac{3}{10}$\\
\hline
\pr{Y=0|Z=2,X=1} & $\dfrac{4}{10}$\\
\hline
\pr{Y=0|X=1} & ?\\
\hline
\end{tabular}
	\vspace*{5pt}
\caption{}
	\label{table:table-2}
\end{table}
for the input probabilities.
The desired probability is then obtained from Table \eqref{table:table-2} as

\begin{align}
    \pr{Y=0|X=1} &= P_1\times\dfrac{5}{10} + P_2\times\dfrac{3}{10}+ P_3\times\dfrac{4}{10}\\
    &= \dfrac{2}{7}\times\dfrac{5}{10} + \dfrac{1}{7}\times\dfrac{3}{10} + \dfrac{4}{7}\times\dfrac{4}{10}\\
    &= \dfrac{1}{7} + \dfrac{3}{70} + \dfrac{16}{70}\\
    &= \dfrac{29}{70}
\end{align}
Hence, the probability of drawing a white ball from bag B is $\dfrac{29}{70}$.

\end{document}
